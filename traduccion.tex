 

\documentclass[final,5p, 12pt,authoryear]{elsarticle}
\usepackage{graphicx}
%\usepackage{fancy}
\usepackage{times}
\usepackage{multirow}
\usepackage{longtable}

% \usepackage{enumitem}
\usepackage{float}
% \usepackage{slashbox}
\usepackage{natbib}
\usepackage{times}
%\usepackage{subfigure}
%\usepackage{ucs}
\usepackage[utf8]{inputenc}
\usepackage[T1]{fontenc}

\usepackage[pdftex]{hyperref}
\usepackage{multicol}
\hypersetup{hyperindex,citecolor=blue,colorlinks=true,linkcolor=blue,bookmarks=true}  
%\pagestyle{plain}
%\pagestyle{fancyhdr}
\title{Transporte cooperativo en hormigas (Hymenoptera: Formicidae) y en todas partes}
\author[rvt]{Tomer J. Czaczkes}
\ead{tomer.czaczkes@gmail.com}

\author[rvt]{Francis L W Ratnieks}
\ead{cosa@gmail.com}
\address[rvt]{Laboratorio de apicultura e insectos sociales, Facultad de Ciencias de la Vida, Universidad de Sussex}
\begin{document}

\begin{frontmatter}
\begin{abstract}
  El transporte cooperativo, definido como varios individuos moviendo al mismo tiempo un objeto, ha aparecido muchas veces en las hormigas, pero es raro en otros animales.
\end{abstract}
\begin{keyword}
  Transporte cooperativo \sep organización \sep forrageo \sep recolección en grupo \sep hormigas \ revisión \sep equipos
\end{keyword}


\end{frontmatter}

\section*{Introducción}
\label{sec:introduccion}

En los últimos cien años, el número de habilidades consideradas únicamente humanas ha disminuido. Por ejemplo, un sentido de la justicia o aversión a la desigualdad ha sido demostrado en monos y perros.

\end{document}




%%% Local Variables: 
%%% mode: latex
%%% TeX-master: t
%%% End: 
